%TCIDATA{LaTeXparent=0,0,relatorio.tex}



\chapter{Conclusões}

\label{CapConclusoes}

O projeto desenvolvido até aqui é uma continuação do que foi desenvolvido por Rédytton \cite{redytton}; porém, muitos avanços foram realizados. Em primeiro lugar, foram melhorados diversos aspectos relativos ao que foi previamente realizado, como por exemplo:
\begin{itemize}
\item Foi efetuada uma calibração da câmera, de forma a se pegar os dados da trajetória de fundo, representada pela bolinha;
\item Foi calculado um fator de conversão entre a unidade de velocidade reconhecida pelo RSLogix e uma unidade de velocidade derivada do SI --- milímetros por segundo;
\item Foi adotado o texto estruturado para se desenvolver as estratégias de controle a serem realizadas pelo CLP, uma vez que o texto é facilmente modificável para novas trajetórias, o que não ocorre com linguagens gráficas como o \textit{ladder};
\item Foi modificado o protocolo de comunicação do CLP com o computador. O novo protocolo, Ethernet/IP, é consideravelmente mais rápido que uma comunicação baseada em RS-232, o que resulta em um tempo menor entre um teste e outro;
\item Um último avanço foi o uso do OPC. Dessa forma, uma estrutura complexa como o preditor de Smith pode ser programada em uma linguagem mais rica em termos de ferramentas do que o texto estruturado --- no caso, foi utilizada a linguagem Python com o módulo OpenOPC \cite{OpenOPC}.
\end{itemize}

No início do trabalho, não havia nenhum conhecimento prévio dos autores sobre programação em tempo real utilizando CLPs; os conhecimentos de programação CLP em geral eram bastante limitados. Além disso, nenhum conhecimento prévio da bancada existia também. No estágio atual, pode-se dizer que há relevante facilidade em se trabalhar com a bancada, apesar de não se ter entrado em detalhes de como funciona o servomotor, \textit{drive} e outros componentes.

O foco do projeto era desenvolver uma estrutura de controle não clássico para lidar com um sistema de ordem infinita, representado pelo conjunto carrinho, bolinha e barbante. Dentro dessa abordagem, procurou-se validar, por meio de um extenso tratamento matemático, o experimento de forma a ele representar, confiavelmente, uma operação de re-entrada com um \textit{riser} real. Dada a estratégia de redução modal de forma a se obter um espaço de estados com uma ordem tal que fosse razoavelmente elementar desenvolver uma estratégia de controle simples, como o de realimentação de estados com canal integral, buscou-se, através do uso do preditor de Smith com Filtro de Kalman, controlar a planta de tal forma que a trajetória de fundo real acompanhasse sua referência com erro mínimo, mesmo na presença de perturbações.

Embora os resultados experimentais não terem sido ideais, pode-se notar que o uso do preditor de Smith é um grande avanço em relação ao teste com malha aberta no que concerne ao acompanhamento da trajetória; porém, considerando todas as fontes de erro oferecidas pela planta em questão --- posicionamento da câmera, perdas de comunicação entre câmera, CLP e computador, atrito na esteira do carrinho, conversão de velocidades, atraso devido à comunicação (\textit{overhead}), entre outros fatores ---, nota-se que os resultados foram satisfatórios; em outras palavras, pode-se dizer que é válido o uso do preditor de Smith como estratégia de controle para uma operação de re-entrada de \textit{risers}, tendo como base os resultados encontrados.


\section{Perspectivas Futuras}
Embora os resultados apresentados até aqui sejam satisfatórios, há muito a ser feito. Além do foco em reduzir o erro em muitos dispositivos da planta, particularmente na câmera, há outros pontos de melhoria no projeto, como por exemplo:
\begin{itemize}
\item Utilizar uma estratégia de calibração da câmera melhor --- a atual não permite excursões maiores que 50 cm;
\item Estudar outras formas de controle para o \textit{riser}, e verificar o desempenho das mesmas em relação ao preditor de Smith;
\item Procurar melhorar a programação, otimizando os algoritmos desenvolvidos e utilizando ferramentas para acelerar os programas já feitos (por exemplo, o uso de \textit{threads}, já feito neste projeto). A razão de se acelerar os programas que fazem a comunicação com por meio do OPC é impedir que seus custos de execução e \textit{overheads} interfiram no tempo de amostragem, comprometendo os resultados;
\item Fazer testes mais minuciosos com as estruturas já desenvolvidas, e utilizar Filtros de Kalman diferentes (por exemplo, filtros adaptativos). O objetivo aqui é atenuar ainda mais as perturbações que venham a ocorrer, visto que um \textit{riser} real é constantemente atingido por forças externas ao controle, como correntes maritmas e eventuais colisões com outros corpos.
\end{itemize}
