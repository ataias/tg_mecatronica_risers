%TCIDATA{LaTeXparent=0,0,relatorio.tex}



\chapter{Conclusões}

\label{CapConclusoes}

No início do trabalho, não havia nenhum conhecimento prévio dos autores sobre programação em tempo real utilizando CLPs, nem sobre programação CLP em geral. Nenhum conhecimento prévio da bancada existia também. No estágio atual, pode-se dizer que há relevante facilidade em se trabalhar com a bancada, apesar de não se ter entrado em detalhes de como funciona o servomotor, \textit{drive} e outros componentes.

O foco principal era fechar a malha, o que foi possível, conforme se observa na seção \ref{malhafechadaSection}. Agora, pode-se testar mais técnicas de controle em malha fechada com a câmera, o que não foi feito antes nesta bancada. Apesar do resultado de um controlador P não ser ideal, é o verdadeiro início da validação de controladores para \textit{risers} com esta bancada, pois na realidade \textit{ROVs} com câmeras são utilizados, o que justifica também ser utilizada uma câmera aqui.


\section{Perspectivas Futuras}

Pra o Trabalho de Graduação 2, objetiva-se utilizar uma técnica de controle mais rebuscada, utilizando-se o projeto de Fabrício et al \cite{fabricioIFAC}. Nele, o modelo é levado em consideração através de redução modal, que o deixa mais simples em questão de custo computacional. Após verificar-se se a mencionada técnica de controle funciona bem mesmo com perturbações, pretende-se analisar outras literaturas na área de controle preditivo e tentar projetar um controlador novo que funcione melhor que o anterior, em algum aspecto.
