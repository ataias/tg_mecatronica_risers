%TCIDATA{LaTeXparent=0,0,relatorio.tex}
\chapter{Fundamentos\label{chap:FundamentacaoMatematica}}

% Resumo opcional. Comentar se não usar.
\resumodocapitulo{Resumo opcional.}

\section{Equações Governantes}

Estruturas submarinas tais como os \textit{risers} são esbeltas e tem um alto módulo de cisalhamento. Portanto, a simplificação de Euler-Bernoulli para vigas é utilizada para propósitos de modelagem. O deslocamento de interesse é o horizontal e o \textit{riser} está sob a ação de forças hidrodinâmicas externas e de tração. A equação diferencial parcial para a variável deslocamento, $\Upsilon$, é dada por \begin{align}
	m_s \frac{\partial^2 \Upsilon}{\partial t^2} &= -E J	\frac{\partial^4 \Upsilon}{\partial z^4} + \frac{\partial}{\partial z}\left(T(z) \frac{\partial \Upsilon}{\partial z}\right) + F_n(z,t),
\end{align} na qual $m_s$ é a densidade linear do tubo, $E$ é o módulo de Young e $J$ é o segundo momento de inércia do \textit{riser}. $T(z)$ descreve as forças de tração ao longo do comprimento do \textit{riser}. $F_n(z,t)$ é a força resultante externa \cite{fabricioIFAC}.

As únicas forças externas atuando no \textit{riser} são hidrodinâmicas, exceto nas extremidades do topo e do fundo, nas quais forças de reação seguem condições de contorno. A equação de Morison descreve a força externa resultante: \begin{align}
	F_n(z,t) &= -m_f \frac{\partial^2 \Upsilon}{\partial t^2} - \mu\left|\frac{\partial \Upsilon}{\partial t}\right|\frac{\partial \Upsilon}{\partial t},
\end{align} na qual $m_f$ é a massa do fluido adicionado e $mu$ é o coeficiente de arrasto. Seja $m = m_s + m_f$ 

\section{Controle}
\paragraph{} Técnicas de controle simples devem ser introduzidas de forma a se compreender o objetivo deste trabalho, que é do posicionamento do \textit{riser} por meio de controle em malha fechada. Primeiramente, apresenta-se o controle em malha aberta, cujo diagrama pode ser observado na Figura \ref{mabertatikz}. A variável $r = r(t)$ é a referência do sistema. Neste tipo de controle, a saída não é realimentada na entrada. Desta forma, este tipo de controle não requer sensores, pois somente dá uma referência de entrada que a planta deve seguir. Caso haja erros para seguir a trajetória, eles não poderão ser compensados e é um controle mais recomendado quando o sistema é preciso e há pouca ou nenhuma perturbação. No entanto, este não é o caso do \textit{riser}, pois o movimento das águas no leito oceânico perturba o tubo, causando erros na posição final desejada.

\tikzstyle{block} = [draw, fill=blue!20, rectangle, 
minimum height=3em, minimum width=6em]
\tikzstyle{sum} = [draw, fill=blue!20, circle, node distance=1cm]
\tikzstyle{input} = [coordinate]
\tikzstyle{output} = [coordinate]
\tikzstyle{pinstyle} = [pin edge={to-,thin,black}]

\begin{figure}[!ht]
	\centering
	% The block diagram code is probably more verbose than necessary
	\begin{tikzpicture}[auto, node distance=2cm,>=latex']
	% We start by placing the blocks
	\node [input, name=input] {};
	\node [block, right of=input] (controller) {Controlador};
	\node [block, right of=controller, pin={[pinstyle]above:Perturbações},
	node distance=3cm] (system) {Planta};
	% We draw an edge between the controller and system block to 
	% calculate the coordinate u. We need it to place the measurement block. 
	\draw [->] (controller) -- node[name=u] {$u$} (system);
	\node [output, right of=system] (output) {};
	
	% Once the nodes are placed, connecting them is easy. 
	\draw [draw,->] (input) -- node {$r$} (controller);
	\draw [->] (system) -- node [name=y] {$y$}(output); 
	\end{tikzpicture}
	\caption{Malha aberta de controle\label{mabertatikz}}
\end{figure}

\paragraph{} Uma forma de se compensar as perturbações do ambiente é realimentando a saída na entrada, calculando a diferença entre a referência e o valor medido. Assim, um valor de erro $e = e(t)$ é obtido e o sistema calcula o sinal $u$ conforme o erro evolui. A Figura \ref{mfechadatikz} mostra um esquema básico deste sistema.

\paragraph{} O controle malha aberta foi anteriormente verificado na referência \cite{redytton}.
\begin{figure}[!ht]
\centering
% The block diagram code is probably more verbose than necessary
\begin{tikzpicture}[auto, node distance=2cm,>=latex']
% We start by placing the blocks
\node [input, name=input] {};
\node [sum, right of=input] (sum) {};
\node [block, right of=sum] (controller) {Controlador};
\node [block, right of=controller, pin={[pinstyle]above:Perturbações},
node distance=3cm] (system) {Planta};
% We draw an edge between the controller and system block to 
% calculate the coordinate u. We need it to place the measurement block. 
\draw [->] (controller) -- node[name=u] {$u$} (system);
\node [output, right of=system] (output) {};
\node [block, below of=u] (measurements) {Medição};

% Once the nodes are placed, connecting them is easy. 
\draw [draw,->] (input) -- node {$r$} (sum);
\draw [->] (sum) -- node {$e$} (controller);
\draw [->] (system) -- node [name=y] {$y$}(output);
\draw [->] (y) |- (measurements);
\draw [->] (measurements) -| node[pos=0.99] {$-$} 
node [near end] {$y_m$} (sum);
\end{tikzpicture}
\caption{Malha fechada de controle\label{mfechadatikz}}
\end{figure}

\section{Redes Utilizadas}
\subsection{DeviceNET}
\paragraph{}DeviceNET é um protocolo de baixo nível da camada de aplicação, voltado a ambientes industriais. É responsável pela interconexão de dispositivos visando ao compartilhamento de dados. Foi desenvolvido pela \textit{Allen-Bradley}, em cima da tecnologia CAN (\textit{Control Area Networking}) desenvolvida pela \textit{Bosch}. Tal rede suporta comunicação entre dispositivos de baixo nível, como sensores e atuadores, e dispositivos de alto nível, como o computador e o CLP. 
\paragraph{}O DeviceNET é uma combinação entre a camada física disponibilizada pelo CAN e um protocolo industrial, o CIP (\textit{Common Industrial Protocol}), que rege as redes industriais em geral. Permite uma rápida configuração entre dispositivos a \textit{byte}, suportanto tanto dispositivos analógicos quanto digitais. Permite velocidades de transmissão de até 500 kbps, sendo bem mais lenta que uma rede Ethernet.
\paragraph{}No experimento, a rede DeviceNET é utilizada para a conexão entre o CLP e os sensores indutivos presentes na planta. Devido à flexibilidade dos sensores e da rede, suas informações podem ser trabalhadas tanto no modo analógico quanto no modo digital. Para fins de detecção de fim-de-curso, entretanto, o modo utilizado é o digital, uma vez que não será preciso determinar a distância entre sensor e carrinho; apenas é necessário verificar se o carrinho está na área de detecção do sensor. O módulo responsável pelo gerenciamento da rede DeviceNET é o 1756-DNB.

\begin{comment}
https://en.wikipedia.org/wiki/DeviceNet
http://ab.rockwellautomation.com/Networks-and-Communications/DeviceNet-Network
http://www.rtaautomation.com/technologies/devicenet/
https://www.odva.org/Portals/0/Library/Publications_Numbered/PUB00122R1_CIP_Brochure_ENGLISH.pdf
\end{comment}

\subsection{Ethernet-IP}
\paragraph{}A rede Ethernet-IP, desenvolvida em meados dos anos 1990, é um tipo de rede Ethernet voltada ao ambiente industrial, seguindo o CIP, assim como o DeviceNET. É uma rede robusta, organizada segundo o modelo OSI de 7 camadas, que permite conexão com dispositivos conectados a redes Ethernet padrão; permite a passagem de dados via pacotes TCP ou UDP; é indicada em aplicações que exigem uma transferência rápida de dados (principalmente, em aplicações de tempo real).
\paragraph{}O uso da rede Ethernet-IP é vantajoso no sentido de que permite a conexão entre vários nós ligados entre si, o que não é possível com o RS-232, por exemplo. Porém, devido ao próprio funcionamento da rede, fica mais difícil obter os dados, pois, ao contrário de uma entrada serial, é necessário lidar com toda uma estrutura baseada no TCP/IP, por exemplo. Além disso, como o Ethernet exige um tamanho mínimo de \textit{frame} para transmissão de dados de cerca de 64 \textit{bytes}, a eficiência da transmissão pode ser afetada.
\paragraph{}No presente experimento, a rede Ethernet-IP é utilizada para receber dados de inspeção da câmera, notadamente a posição da bola de isopor presa ao barbante. Além disso, ela é utilizada na transferência de programas entre o computador e o CLP (através do módulo Ethernet 1756-ENBT/A), uma vez que ela provê uma comunicação mais rápida do que o RS-232.

\begin{comment}
https://en.wikipedia.org/wiki/Industrial_Ethernet
https://en.wikipedia.org/wiki/EtherNet/IP
https://www.odva.org/Technology-Standards/EtherNet-IP/Overview
http://www.rtaautomation.com/technologies/ethernetip/
http://www.rockwellautomation.com/global/products-technologies/integrated-architecture/ethernet-ip.page
http://www.rtaautomation.com/technologies/ethernetip/
\end{comment}

\section{Programação do CLP}
\subsection{Visão geral}

\subsection{Linguagem \textit{ladder}}

\subsection{Texto Estruturado}
