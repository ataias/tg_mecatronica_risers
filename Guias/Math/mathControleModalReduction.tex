%%%%%%%%%%%%%%%%%%%%%%%%%%%%%%%%%%%%%%%%%
% Short Sectioned Assignment
% LaTeX Template
% Version 1.0 (5/5/12)
%
% This template has been downloaded from:
% http://www.LaTeXTemplates.com
%
% Original author:
% Frits Wenneker (http://www.howtotex.com)
%
% License:
% CC BY-NC-SA 3.0 (http://creativecommons.org/licenses/by-nc-sa/3.0/)
%
%%%%%%%%%%%%%%%%%%%%%%%%%%%%%%%%%%%%%%%%%

%----------------------------------------------------------------------------------------
%	PACKAGES AND OTHER DOCUMENT CONFIGURATIONS
%----------------------------------------------------------------------------------------

\documentclass[a4paper,11pt]{scrartcl} % A4 paper and 11pt font size

\usepackage[T1]{fontenc} % Use 8-bit encoding that has 256 glyphs
\usepackage[utf8]{inputenc}
%\usepackage{fourier} % Use the Adobe Utopia font for the document - comment this line to return to the LaTeX default
\usepackage[brazilian]{babel} % English language/hyphenation
\usepackage{amsmath,amsfonts,amsthm} % Math packages

\usepackage{verbatim}
\usepackage{lipsum} % Used for inserting dummy 'Lorem ipsum' text into the template
\usepackage{graphicx}
\usepackage{sectsty} % Allows customizing section commands
\allsectionsfont{\centering \normalfont\scshape} % Make all sections centered, the default font and small caps
\usepackage{float}
\usepackage{fancyhdr} % Custom headers and footers
\pagestyle{fancyplain} % Makes all pages in the document conform to the custom headers and footers
\fancyhead{} % No page header - if you want one, create it in the same way as the footers below
\fancyfoot[L]{} % Empty left footer
\fancyfoot[C]{} % Empty center footer
\fancyfoot[R]{\thepage} % Page numbering for right footer
\renewcommand{\headrulewidth}{0pt} % Remove header underlines
\renewcommand{\footrulewidth}{0pt} % Remove footer underlines
\setlength{\headheight}{13.6pt} % Customize the height of the header

\numberwithin{equation}{section} % Number equations within sections (i.e. 1.1, 1.2, 2.1, 2.2 instead of 1, 2, 3, 4)
\numberwithin{figure}{section} % Number figures within sections (i.e. 1.1, 1.2, 2.1, 2.2 instead of 1, 2, 3, 4)
\numberwithin{table}{section} % Number tables within sections (i.e. 1.1, 1.2, 2.1, 2.2 instead of 1, 2, 3, 4)

\setlength\parindent{0pt} % Removes all indentation from paragraphs - comment this line for an assignment with lots of text

%----------------------------------------------------------------------------------------
%	TITLE SECTION
%----------------------------------------------------------------------------------------

\newcommand{\horrule}[1]{\rule{\linewidth}{#1}} % Create horizontal rule command with 1 argument of height

\title{
\normalfont \normalsize
\textsc{Universidade de Brasília} \\ [25pt] % Your university, school and/or department name(s)
\horrule{0.5pt} \\[0.4cm] % Thin top horizontal rule
\huge Implementação de Controle com Redução Modal \\ % The assignment title
\horrule{2pt} \\[0.5cm] % Thick bottom horizontal rule
}

\author{Ataias Pereira Reis \\ Emanuel Pereira Barroso Neto} % Your name

\date{\normalsize\today} % Today's date or a custom date

\usepackage{beramono}
\usepackage{listingsutf8}
\usepackage[usenames,dvipsnames]{xcolor}

%%
%% Julia definition (c) 2014 Jubobs
%%
\lstdefinelanguage{Julia}%
  {morekeywords={abstract,break,case,catch,const,continue,do,else,elseif,%
      end,export,false,for,function,immutable,import,importall,if,in,%
      macro,module,otherwise,quote,return,switch,true,try,type,typealias,%
      using,while},%
   sensitive=true,%
   alsoother={\$},%
   morecomment=[l]\#,%
   morecomment=[n]{\#=}{=\#},%
   morestring=[s]{"}{"},%
   morestring=[m]{'}{'},%
}[keywords,comments,strings]%

\lstset{%
    language         = Julia,
    basicstyle       = \ttfamily,
    keywordstyle     = \bfseries\color{blue},
    stringstyle      = \color{magenta},
    commentstyle     = \color{ForestGreen},
    showstringspaces = false
}

\lstset{
	language = Python,
	basicstyle=\ttfamily\small,
    numberstyle=\footnotesize,
    numbers=left,
    backgroundcolor=\color{gray!10},
    frame=single,
    tabsize=2,
    rulecolor=\color{black!30},
    title=\lstname,
    escapeinside={\%*}{*)},
    breaklines=true,
    breakatwhitespace=true,
    framextopmargin=2pt,
    framexbottommargin=2pt,
    extendedchars=true,
    inputencoding=utf8,
    literate={á}{{\'a}}1 {ã}{{\~a}}1 {é}{{\'e}}1 {ç}{{\c{c}}}1 {â}{{\^a}}1 {õ}{{\~o}}1 {ú}{{\'u}}1 {ó}{{\'o}}1 {í}{{\'i}}1 {Í}{{\'I}}1 
}

\usepackage{hyperref}
\hypersetup{pdftex,colorlinks=true,allcolors=blue}
\usepackage{hypcap}

\begin{document}
\maketitle % Print the title

%----------------------------------------------------------------------------------------
%	PROBLEM 1
%----------------------------------------------------------------------------------------

\section{Introdução}
\paragraph{} O objetivo deste documento é apresentar os procedimentos necessários para implementar o método de controle apresentado no artigo ``\textit{Modal Reduction Based Tracking Control for Installation of Subsea Equipments}'', desenvolvido por Fabrício et al, em um controlador industrial da Rockwell. Nem todos os detalhes estão presentes no artigo, o que torna difícil simplesmente lê-lo e realizar o sistema. Algumas modificações no controle serão feitas com base no trabalho do aluno de mestrado  \href{mailto:rafael.domenici@hotmail.com}{Rafael Simões <rafael.domenici@hotmail.com>}.

\section{Equações Governantes}
%TODO: Organizar constantes em uma subseção e especificar exatamente o valor de cada uma daquelas que temos interesse, de forma a se definir corretamente os valores das constantes "a", "b_k", "c", "d_k" e "e_k"
\paragraph{} Para o riser, a Equação \ref{equacaoMorison} representa o deslocamento horizontal $\Upsilon(z,t)$ do tubo --- um barbante, no caso da bancada de laboratório --- sob a ação de forças hidrodinâmicas externas $F_n(z,t)$ (força linear com unidade N/m) e tração $T(z)$ (unidade N): \begin{align}
	m_s \frac{\partial^2 \Upsilon}{\partial t^2} &= -EJ \frac{\partial^4 \Upsilon}{\partial z^4} + 	\frac{\partial}{\partial z}\left(T(z)\frac{\partial \Upsilon}{\partial z}\right)+F_n(z,t) \label{equacaoMorison}
\end{align}

\paragraph{} Antes de prosseguir, é importante definir termos desta equação: \begin{itemize}
	\item $m_s$ é a massa linear do barbante (densidade linear, kg/m)
	\item $E$ é o módulo de Young do barbante e ele é desconhecido
	\item $J$ é o segundo momento de área e representa a resistência do barbante à flexão. O barbante não apresenta tal resistência, daí $J=0$
	\item $T(z)$ é a força de tração e é dada por \[T(z) = \left(m_b+z m_s\right)g,\] sendo $m_b$ a massa da bolinha (kg), $m_s = m_{\textrm{barbante,kg}}/L$, sendo $L$ o comprimento do barbante, $z$ a posição vertical a partir do carrinho e $g$ é a força da gravidade.
\end{itemize}

\paragraph{} A força externa resultante, $F_n(z,t)$, é dada por \begin{align}
	F_n(z,t) &= -m_{fbar} \frac{\partial^2 \Upsilon}{\partial t^2} - \mu \left|\frac{\partial \Upsilon}{\partial t}\right|\frac{\partial \Upsilon}{\partial t}\label{forceN},
\end{align} na qual $\mu$ é o coeficiente de arrasto (unidade 1/s) e $m_{fbar}$ é a massa do fluido adicionado, que será posteriormente pormenorizada. Fazendo $m = m_s + m_{fbar}$ e substituindo a Equação \ref{forceN} na \ref{equacaoMorison}, obtém-se: \begin{align}
	\frac{\partial^2 \Upsilon}{\partial t^2} &= -\frac{EJ}{m}\frac{\partial^4 \Upsilon}{\partial z^4} + \frac{\partial}{\partial z}\left(\frac{T(z)}{m}\frac{\partial \Upsilon}{\partial z}\right) - \frac{\mu}{m}\left|\frac{\partial \Upsilon}{\partial t}\right|\frac{\partial \Upsilon}{\partial t}
\end{align}

\subsection{Constantes}
\begin{table}[!ht]
	\centering
	\begin{tabular}{|l|c|c|c|}
		\hline
		\textbf{Significado} & \textbf{Símbolo} & \textbf{Valor} & \textbf{Unidade}\\ \hline \hline
		Massa & $m_{bar}$ & 0.492 & g\\ \hline
		Comprimento & $L$ & 0.82 & m \\ \hline
		Massa linear & $m_s$ & 0.6 & g/m\\ \hline
		Raio & $r_{bar}$ & 1 & mm\\ \hline
		Densidade & $\rho_{bar}$ & 191 & kg/m$^3$\\ \hline
	\end{tabular}
	\caption{Constantes do barbante}
\end{table}

\begin{table}[!ht]
	\centering
	\begin{tabular}{|l|c|c|c|}
		\hline
		\textbf{Significado} & \textbf{Símbolo} & \textbf{Valor} & \textbf{Unidade}\\ \hline \hline
		Massa & $m_{b}$ & 0.492 & g\\ \hline
		Raio & $r_{b}$ & 15.3 & mm\\ \hline
		Coeficiente de inércia & $C_m$ & 1.2 & - \\ \hline
		Coeficiente de arrasto & $C_d$ & 0.6 & - \\ \hline
		Volume & $V_b$ & $\frac{4}{3}\pi r_b^3$ & $\textrm{m}^3$ \\ \hline
		Área da seção transversal & $A_b$ & $\pi r_b^2$ & m$^2$\\ \hline
	\end{tabular}
	\caption{Constantes da bolinha de isopor}
\end{table}

\paragraph{} Como visto anteriormente, $m = m_s + m_{fbar}$. A massa linear $m_{fbar}$ do fluido adicionado ao redor do barbante é dada por \begin{align}
	m_{fbar} &=  2 \pi r_{bar}^2 \rho_{\mathrm{ar}}\nonumber\\
	&= 0.00770\;\mathrm{g/m}.
\end{align} 

\paragraph{} Já que $m_{fbar} \ll m_s$, consideraremos $m = m_s$ nos cálculos. Em relação à massa $m_{fb}$ do fluido adicionado ao redor da bolinha de isopor, ela é dada por\begin{align}
	m_{fb} &= 1.2 V_{b} \rho_{\mathrm{ar}}\nonumber\\
	&= 1.2 \left(\frac{4}{3}\pi r_b^3\right)\rho_{\mathrm{ar}}\nonumber\\
	&= 0.0220\;\textrm{g}
\end{align}

\paragraph{} $m_{fb} \ll m_{b}$ também, de forma que os cálculos consideraram $m' = m_b$.

\paragraph{} Nota-se que o barbante pesa mais que o isopor, o que faz com que a tração não seja principalmente devida pela bolinha, mas sim pelo barbante. Neste caso, não se utiliza um valor médio para $T(z)$ como no artigo do Fabrício, mas ainda se pode usar um valor médio para as constantes $\tau$ e $\tau'$, que substituem o termo $\frac{\mu}{m}\left|\frac{\partial \Upsilon}{\partial t}\right|$ para o barbante e para a bolinha, respectivamente. Essas constantes são definidas de acordo com a trajetória prevista, uma vez que a velocidad média depende dessa trajetória. Já levando em conta um valor médio para $\left|\frac{\partial \Upsilon}{\partial t}\right|$, tem-se \begin{align}
	\frac{\partial^2 \Upsilon}{\partial t^2} &= -\frac{EJ}{m}\frac{\partial^4 \Upsilon}{\partial z^4} + \frac{\partial}{\partial z}\left(\frac{T(z)}{m}\frac{\partial \Upsilon}{\partial z}\right) - \tau\frac{\partial \Upsilon}{\partial t}\label{EquacaoComTau}
	\end{align}

\paragraph{} Antes de prosseguirmos para a discretização e obtenção das matrizes em espaço de estados, é importante pensar nas condições de contorno. No topo, $z=L$, tem-se $\Upsilon(L,t)=u(t)$, ou seja, o carrinho se move conforme uma trajetória $u(t)$ definida. Neste mesmo ponto, $\frac{\partial\Upsilon}{\partial z}(L,t) = 0$. Para a ponta na qual a carga está situada, $z=0$, tem-se $\frac{\partial\Upsilon}{\partial z}(0,t) = \frac{F_L}{T}$, sendo $F_L$ a força aplicada pela ponta do riser na carga. \textbf{(Outra coisa que confundi, eu entendi $u(t)$ sendo uma trajetória, pois $\Upsilon$ é deslocamento, mas no artigo do Fabrício está escrito em uma momento que é uma força)}.

\subsection{Discretização}
\paragraph{} De forma a se realizar o controle proposto, o sistema deve ter um espaço de estados finito. Para isso, aplica-se o método de diferenças finitas na coordenada $z$ de maneira a se aproximar a EDP governante em um número finito de EDOs. No espaço discreto, a equação do $k$-ésimo elemento é dada por \begin{align}
	\frac{d^2\Upsilon_k}{dt^2} &= -\frac{EJ}{m l^4}\left(\Upsilon_{k-2} - 4\Upsilon_{k-1}+6\Upsilon_{k}-4\Upsilon_{k+1}+\Upsilon_{k+2}\right)\nonumber\\
	&+ \frac{T_0+mg(k-1)l}{m l^2}\left(\Upsilon_{k-1}-2\Upsilon_{k} + \Upsilon_{k+1}\right)+g\frac{-\Upsilon_{k-1}+\Upsilon_{k+1}}{2l}-\tau\frac{d\Upsilon_k}{dt},
\end{align} sendo $N$ o número de pontos de discretização e $l$ a distância entre dois pontos de discretização ($l = L/N$).

\paragraph{} Note que $k\in \mathbb{N}:2\le k \le N-1$, pois um dos extremos é a bolinha e a equação do pêndulo rege seu movimento enquanto que a outra ponta se aplica uma condição de contorno. O que aconteceria quando $k=2$ e se precisasse de $\Upsilon_{k-2}$? Para nosso experimento, $J=0$ e esse problema não ocorre. Caso se façam testes com um valor de $J\neq 0$, teríamos de resolver esse problema primeiro.

\paragraph{} Para simplificar, definem-se as constantes \begin{align}
	a &= -\frac{EJ}{m l^4}\\
	b_k &= \frac{T_0 + mg(k-1)l}{m l^2},\; k\ge 2\\
	c &= \frac{g}{2l}\\
	d_k &= b_k - c,\; k\ge 2\\
	e_k &= b_k + c,\; k\ge 2
\end{align}

\paragraph{} A meu ver, a melhor forma de se analisar como as matrizes do sistema ficarão é expandir o sistema para casos com $N$ pequeno e ver o que está ocorrendo. Observe que $a=0$ para o barbante, o que simplifica os próximos passos.

\paragraph{} Para o caso $N=6$, tem-se \begin{align}
\mathbf{x} &= \left(\Upsilon_1\;\Upsilon_2\;\Upsilon_3\;\Upsilon_4\;\Upsilon_5\;\Upsilon_6\;\dot{\Upsilon}_1\;\dot{\Upsilon}_2\;\dot{\Upsilon}_3\;\dot{\Upsilon}_4\;\dot{\Upsilon}_5\;\dot{\Upsilon}_6\right)^T 	\\
u &= \Upsilon(L,t) = \Upsilon_7\label{ufor6}\\
y &= \Upsilon(0,t) = \Upsilon_1\label{yfor6}
 \end{align} e as equações são \begin{align}
 	\ddot{\Upsilon}_2 &=  b_2\left(\Upsilon_{1}-2\Upsilon_{2} + \Upsilon_{3}\right)+c(-\Upsilon_1 + \Upsilon_3)-\tau \dot{\Upsilon}_2\nonumber \\
 	&= d_2\Upsilon_1 - 2b_2 \Upsilon_2 + e_2\Upsilon_3 - \tau \dot{\Upsilon}_2\\
 	\ddot{\Upsilon}_3 &=  b_3\left(\Upsilon_{2}-2\Upsilon_{3} + \Upsilon_{4}\right)+c(-\Upsilon_2 + \Upsilon_4)-\tau \dot{\Upsilon}_3\nonumber \\
 	&= d_3\Upsilon_2 - 2b_3 \Upsilon_3 + e_3\Upsilon_4 - \tau \dot{\Upsilon}_3\\
 	\ddot{\Upsilon}_4 &=  b_4\left(\Upsilon_{3}-2\Upsilon_{4} + \Upsilon_{5}\right)+c(-\Upsilon_3 + \Upsilon_5)-\tau \dot{\Upsilon}_4 \nonumber\\
 	&= d_4\Upsilon_3 - 2b_4 \Upsilon_4 + e_4\Upsilon_5 - \tau \dot{\Upsilon}_4\\
 	\ddot{\Upsilon}_5 &=  b_5\left(\Upsilon_{4}-2\Upsilon_{5} + \Upsilon_{6}\right)+c(-\Upsilon_4 + \Upsilon_6)-\tau \dot{\Upsilon}_5\nonumber\\
 	&= d_5\Upsilon_4 - 2b_5 \Upsilon_5 + e_5\Upsilon_6 - \tau \dot{\Upsilon}_5\\
 	\ddot{\Upsilon}_6 &=  b_6\left(\Upsilon_{5}-2\Upsilon_{6} + u\right)+c(-\Upsilon_5 + u)-\tau \dot{\Upsilon}_6\nonumber\\
 	&= d_6\Upsilon_5 - 2b_6 \Upsilon_6 + e_6 u - \tau \dot{\Upsilon}_6
 \end{align}

 \paragraph{} A equação para a posição da carga $\Upsilon_1$ leva em conta a massa da bolinha e a força de Morison: \begin{align}
 	m_b \ddot{\Upsilon}_1 &= \frac{m_b g}{l}\left(\Upsilon_2 - \Upsilon_1\right) + \rho_{\mathrm{ar}} C_m V_b \ddot{\Upsilon}_1 - \frac{1}{2}\rho_{\textrm{ar}} C_d A_b \dot{\Upsilon}_1 \left|\dot{\Upsilon}_1\right|,
 \end{align} e, isolando-se $\ddot{\Upsilon}_1$, tem-se \begin{align}
 	\ddot{\Upsilon}_1 &= \frac{m_b g}{m'l}\left(\Upsilon_2 - \Upsilon_1\right)  - \frac{1}{2m'}\rho C_d A_b \dot{\Upsilon}_1 \left|\dot{\Upsilon}_1\right|.
 \end{align} 
 
 \paragraph{} Note que $m' = m_b + \rho_{\textrm{ar}} C_m V_b = m_b + m_{fb} \approx m_b$. Assim, assume-se $m' = m_b$ para os cálculos.

 \paragraph{} Anteriormente, foi apresentada a linearização $\tau$ para o termo $\frac{1}{2m}\rho C_d A\left|\dot{\Upsilon}_k\right|$ do cabo. Assumo que isso também seja necessário para a bola, resultando em um $\tau'$: \begin{align}
 	\ddot{\Upsilon}_1 &= b_1\left(-\Upsilon_1 + \Upsilon_2\right) - \tau'\dot{\Upsilon}_1,
 \end{align} com \begin{equation}
 	b_1 = \frac{m_b g}{m'l} = \frac{g}{l}.
 \end{equation}

 \paragraph{} Desta forma, pode-se definir o sistema linear em forma matricial \begin{align}
 	\mathbf{\dot{x}} &= \left[\begin{array}{cccccccccccc}
 		0 & 0 & 0 & 0 & 0 & 0 & 1 & 0 & 0 & 0 & 0 & 0\\
 		0 & 0 & 0 & 0 & 0 & 0 & 0 & 1 & 0 & 0 & 0 & 0\\
 		0 & 0 & 0 & 0 & 0 & 0 & 0 & 0 & 1 & 0 & 0 & 0\\
 		0 & 0 & 0 & 0 & 0 & 0 & 0 & 0 & 0 & 1 & 0 & 0\\
 		0 & 0 & 0 & 0 & 0 & 0 & 0 & 0 & 0 & 0 & 1 & 0\\
 		0 & 0 & 0 & 0 & 0 & 0 & 0 & 0 & 0 & 0 & 0 & 1\\
 		-b_1 & b_1 & 0 & 0 & 0 & 0 & -\tau' & 0     & 0 & 0 & 0 & 0\\
 		d_2 & -2b_2  & e_2  & 0  & 0 & 0 &  0    & -\tau & 0 & 0 & 0 & 0\\
 		0 & d_3 & -2b_3  & e_3  & 0  & 0 & 0 &  0    & -\tau & 0 & 0 & 0\\
 		0 & 0 & d_4 & -2b_4  & e_4  & 0  & 0 & 0 &  0    & -\tau & 0 & 0\\
 		0 & 0 & 0 & d_5 & -2b_5  & e_5  & 0  & 0 & 0 &  0    & -\tau & 0\\
 		0 & 0 & 0 & 0 & d_6 & -2b_6  & 0  & 0 & 0 &  0    & 0   &-\tau\\
 	\end{array}\right]\mathbf{x} + \left[\begin{array}{c}
	0\\	0\\	0\\	0\\	0\\ 0\\ 0\\ 0\\ 0\\ 0\\ 0\\ e_6
\end{array}
\right]u
 \end{align} que pode ser representado concisamente como \begin{align}
 	\mathbf{\dot{x}} &= \left[\begin{array}{cc}
	\mathbf{0}_{6\times 6} & \mathbf{I}_{6\times 6}\\
	\mathbf{M}_{6\times 6} & \mathbf{L}_{6\times 6}\\
\end{array}\right] \mathbf{x} + \left[\begin{array}{c}
	\mathbf{0}_{11\times 1}\\ e_6\\
\end{array} \right]u
 \end{align}

 Para o caso de uma discretização com $N$ pontos, tem-se \begin{align}
 	\mathbf{\dot{x}} &= \left[\begin{array}{cc}
	\mathbf{0}_{N\times N} & \mathbf{I}_{N\times N}\\
	\mathbf{M}_{N\times N} & \mathbf{L}_{N\times N}\\
\end{array}\right] \mathbf{x} + \left[\begin{array}{c}
	\mathbf{0}_{2N-1\times 1}\\ e_N\\
\end{array} \right]u\\
y &= \left[\begin{array}{cc}
	1 & \textbf{0}_{1\times 2N-1}
\end{array}\right]\textbf{x}
 \end{align}
 
 \subsection{Código para calcular matrizes}
%
% \paragraph{} Esta matriz fornece um padrão que pode ser utilizado para se criar um algoritmo que gera as matrizes $\mathbf{A}$ e $\mathbf{B}$ automaticamente, sendo $\mathbf{\dot{x}} = \mathbf{A}\mathbf{x}+\mathbf{B}u$.
%
% \lstinputlisting[language=Julia,caption={Código para gerar matrizes A, B e C},label=generateABC]{code/GenerateABC.jl}
\lstinputlisting[language=Julia,caption={Código para gerar matrizes A, B e C},label=generateABC]{code/ModalReduction.jl}


\section{Uma Estratégia de Redução da Ordem do Modelo}
\paragraph{} A maior parte da teoria clássica de controle lida com sistemas representados por um pequeno número de variáveis de estado. Portanto, uma forma de aplicar métodos clássicos de controle da literatura para sistemas de parâmetros distribuídos discretos é por meio de uma redução da ordem do modelo.

\paragraph{} Tal redução do modelo será feita em duas etapas: primeiro, uma transformação modal é aplicada nas equações originais do espaço de estados, resultando em uma nova representação em variáveis modais. Nesta forma, o sistema pode ser visto como um conjunto de subsistemas dissociados em paralelo, cuja influência na saída pode ser calculada individualmente. Então, os subsistemas com os maiores ganhos estáticos são escolhidos para criar um modelo de ordem reduzida.

\subsection{Decomposição Modal}

\paragraph{} Primeiro, deve-se obter os autovalores do espaço de estados do \textit{riser}. Observa-se que eles são sempre distintos entre si, uma condição suficiente para a diagonalização da matriz do espaço de estados. Assim, calcula-se a matriz modal \textbf{T}, cuja i-ésima coluna é o i-ésimo autovetor do sistema: \begin{align}
	\mathbf{T} &= \left(\;\mathbf{v_1}\;|\;\mathbf{v_2}\;|\;\ldots\;|\;\mathbf{v_{2N}}\;\right)_{1\times 2N}
\end{align}

\paragraph{} A matriz $\mathbf{T}$ provavelmente tem valores complexos. Isso é um problema para a representação em espaço de estados e sua simulação. A solução é criar uma matriz $\mathbf{\tilde{T}}$ que tenha só números reais. Antes de explicar como criá-la, lembre que os autovalores complexos sempre aparecem em pares conjugados já que a matriz $\mathbf{A}$ só tem valores reais. Quando a primeira coluna de um autovetor de um par complexo conjugado for encontrada, a coluna respectiva de $\mathbf{\tilde{T}}$ será sua parte real. A segunda coluna desse par complexo conjugado será a parte imaginária da coluna de $\mathbf{T}$.

\paragraph{} A matriz $\mathbf{\tilde{T}}$ é utilizada para uma transformação de similaridade no sistema original: \begin{align}
	\mathbf{A_M} &= \mathbf{\tilde{T}}^{-1}\mathbf{A}\mathbf{\tilde{T}},\\
	\mathbf{x_M} &=\mathbf{\tilde{T}}^{-1}\mathbf{x},	\\
	\mathbf{B_M} &= \mathbf{\tilde{T}}^{-1}\mathbf{B},\textrm{ e}\\
	\mathbf{C_M} &=\mathbf{C}\mathbf{\tilde{T}}.
\end{align}

\paragraph{} O sistema transformado, denotado pelo subscrito $\mathbf{M}$, é mais adequado à análise. $\mathbf{A_M}$ é uma matriz diagonal, com seus autovalores explícitos, e permitindo o desacoplamento do sistema original em $N$ subsistemas de segunda ordem formados por pares de autovalores reais ou complexo-conjugados.

%TODO Escrever o código para criar a matriz T e fazer a transformação de similaridade

\subsection{Redução Modal}
\paragraph{} Neste estágio, procura-se determinar quais dos subsistemas são mais adequados para aproximar o modelo original por meio do cálculo do ganho estático de cada um. Este método depende da predominância de uns poucos autovalores na resposta do sistema, já que altas frequências são muito atenuadas pelas forças hidrodinâmicas e pela suavidade da entrada.
\paragraph{} Os subsistemas selecionados são combinados em um modelo reduzido \begin{align}
	\mathbf{\dot{z}} &= \mathbf{A_R}\mathbf{z}+\mathbf{B_R}u\\
	y &= \mathbf{C_R}\mathbf{z}+\mathbf{D_R}u
\end{align} cuja ordem é escolhida considerando o custo-benefício entre a acurácia da dinâmica reduzida e a simplicidade da estrutura de controle exigida. Além disso, o sistema reduzido deve compensar o ganho estático perdido nos autovalores desconsiderados. Isto é feito por meio de uma matriz de transferência direta $\mathbf{D_R}$, que é a diferença dos ganhos dos sistemas original e reduzido: \begin{align}
	\mathbf{D_R}&=\mathbf{C}\mathbf{A^{-1}}\mathbf{B}-\mathbf{C_R}\mathbf{A_R^{-1}}\mathbf{B_R}
\end{align}

\paragraph{} O subsistemas são de ordem 1 ou 2 dependendo se o autovalor é real ou um par complexo conjugado. %TODO: Explicar algoritmo para calcular os ganhos estáticos e escolher os autovalores mais relevantes

\paragraph{} A matriz de transferência direta $\mathbf{D_R}$ introduz novas dinâmicas: uma saída não-nula que não leva em conta o atraso de propagação da entrada e um ganho em altas frequências. Conforme mostrado por Fortaleza (2009), podemos refinar o modelo reduzido introduzindo um atraso de entrada $\epsilon$ que minimiza a transferência direta e garante dinâmica nula para $t < \epsilon$: \begin{align}
\begin{array}{lll}
	\mathbf{\dot{z}} &=& \mathbf{A_R}\mathbf{z}+\mathbf{B_D}u(t-\epsilon)\\
	y &=& \mathbf{C_R}\mathbf{z}+\mathbf{D_D}u(t-\epsilon) \label{novoModeloReduzido}
\end{array}
\end{align} sendo \begin{align}
	\mathbf{B_D} &= \mathbf{A_M}\left(e^{\epsilon\mathbf{A_M}}\right)\mathbf{A_M^{-1}}\mathbf{B_M}\\
	\mathbf{D_D} &= \mathbf{C_M}\left(e^{\epsilon\mathbf{A_M}} - \mathbf{I}\right)\mathbf{A_M^{-1}}\mathbf{B_M} + \mathbf{D_M}
\end{align}

\paragraph{} O novo modelo reduzido (\ref{novoModeloReduzido}) é tal que, para uma entrada degrau no instante $t'$, a saída mantém seu valor inicial enquanto $t < t' + \epsilon$. Para $t \ge t' + \epsilon$, ambos os modelos reduzidos produzem a mesma saída. O atraso $\epsilon$ pode ser visto como uma aproximação para o atraso natural de propagação da estrutura.

%TODO: Escrever algoritmo para esta parte também
\begin{comment}
\section{Projeto de Controle}
\subsection{Planejamento Offline de Trajetória}
\paragraph{} A dinâmica do modelo de ordem reduzida (\ref{novoModeloReduzido}) é comparada com aquela do sistema original. Escolhendo uma redução para ordem 4, com dois pares de autovalores complexo-conjugados ($\lambda_i,\overline{\lambda}_i$, com $\lambda_i = \sigma_i + j w_i$), a equação do sistema em espaço de estados é \begin{align}
	\left[\begin{array}{c}
	\dot{z}_1\\ \dot{z}_2\\ \dot{z}_3\\ \dot{z}_4
	\end{array}\right] &= \left[\begin{array}{cccc}
	\sigma_1 & w_1 & 0 & 0\\ -w_1 & \sigma_1 & 0 & 0\\
	0 & 0 & \sigma_2 & w_2\\ 0 & 0 & -w_2 & \sigma_2
\end{array}\right] \left[\begin{array}{c}z_1\\ z_2\\ z_3\\ z_4\end{array}\right] + \left[\begin{array}{c}b_1\\ b_2\\ b_3\\ b_4\end{array}\right]u(t-\epsilon)
\end{align} \begin{align}
	y &= \left[\begin{array}{cccc}
	c_1 & c_2 & c_3 & c_4
\end{array}\right] \left[\begin{array}{c}
	z_1\\ z_2\\ z_3\\ z_4
\end{array}\right] + (d) u(t-\epsilon)
\end{align}

\paragraph{} As constantes $b_i$ e $c_i$ são diferentes das definidas anteriormente. %TODO: Revisar documento evitando redefinição de constantes com mesmo nome

\paragraph{} Normalmente o problema de planejamento de trajetória consiste em encontrar um controle em malha aberta $u^*(t)$, de forma que as variáveis de estado sigam uma trajetória $z^*(t)$. O problema aqui é que $z^*(t)$ carece de interpretação física; ou seja, $z^*(t)$ não se relaciona de maneira clara com nenhuma variável de estado original do \textit{riser}, e, portanto, com a operação de re-entrada. Uma forma de se lidar com isso é lidar com a trajetória da saída \textit{flat} do sistema. %TODO revisar essa parte de flatness

\paragraph{} Para um sistema ser \textit{flat}, é condição suficiente ele ser plenamente controlável, ou seja, a matriz de controlabilidade $K = \left[
\begin{array}{cccc}
B & AB & A^{2}B & A^{3}B
\end{array} \right ]$ deve ser de posto pleno. A partir de $z$, podemos obter uma saída \textit{flat} ($f$), dada por uma combinação linear das variáveis de estado $\left (z_1, z_2, z_3, z_4 \right )$ obtida utilizando-se a última linha de $K^{-1}$:
\begin{align}
	f = \left[\begin{array}{cccc}
	0 & 0 & 0 & 1
	\end{array}\right]K^{-1}z
\end{align} %TODO negritar algumas variáveis
\begin{align}
\label{f_equation}
	f &= \alpha_1 z_1 + \alpha_2 z_2 + \alpha_3 z_3 + \alpha_4 z_4
\end{align}
\paragraph{}Deseja-se que a entrada $u$ dependa apenas de $f$ e suas derivadas temporais. Para isso, é necessário que $f$ seja tantas vezes derivável quanto indica a ordem do sistema. No presente estudo, $f$ deve ser ao menos 4 vezes derivável. Derivando $f$ de acordo com a Equação \ref{f_equation}, temos:
\begin{align}
	\label{f_derivatives}
	\begin{array}{lcl}
	\dot{f} &=& \alpha_5 z_1 + \alpha_6 z_2 + \alpha_7 z_3 + \alpha_8 z_4 \\
	\ddot{f} &=& \alpha_9 z_1 + \alpha_{10} z_2 + \alpha_{11} z_3 + \alpha_{12} z_4 \\
	\dddot{f} &=& \alpha_{13} z_1 + \alpha_{14} z_2 + \alpha_{15} z_3 + \alpha_{16} z_4 \\
	f^{(4)} &=& \alpha_{17} z_1 + \alpha_{18} z_2 + \alpha_{19} z_3 + \alpha_{20} z_4 + \gamma u
	\end{array}
\end{align} %TODO explicar de onde vem o \gamma u
\paragraph{}Das Equações \ref{f_equation} e \ref{f_derivatives}, podemos definir a seguinte matriz $M$, sabendo que
\begin{align}
\label{M_init_equation}
\left (
\begin{array}{c}
f \\ \dot{f} \\ \ddot{f} \\ \dddot{f} \\ f^{(4)}
\end{array}
\right ) = M \left(
\begin{array}{c}
z_1 \\ z_2 \\ z_3 \\ z_4 \\ u
\end{array}
\right)
\end{align}
\paragraph{}Pela Equação \ref{M_init_equation}:
\begin{align}
\label{M_decl_ref}
M = \left (
\begin{array}{ccccc}
	\alpha_{1} & \alpha_{2} & \alpha_{3} & \alpha_{4} & 0 \\
	\alpha_{5} & \alpha_{6} & \alpha_{7} & \alpha_{8} & 0 \\
	\alpha_{9} & \alpha_{10} & \alpha_{11} & \alpha_{12} & 0 \\
	\alpha_{13} & \alpha_{14} & \alpha_{15} & \alpha_{16} & 0 \\
	\alpha_{17} & \alpha_{18} & \alpha_{19} & \alpha_{20} & \gamma
\end{array}
\right )
\end{align}
\paragraph{}Para se obter $u$ em função de uma trajetória $f$, podemos utilizar a Equação \ref{M_init_equation}. Multiplicando os dois membros da equação à esquerda por $M^{-1}$, vem
\begin{align}
\label{M_inv_equation}
\left(
\begin{array}{c}
z_1 \\ z_2 \\ z_3 \\ z_4 \\ u
\end{array}
\right) = M^{-1}\left (
\begin{array}{c}
f \\ \dot{f} \\ \ddot{f} \\ \dddot{f} \\ f^{(4)}
\end{array}
\right )
\end{align}
\paragraph{} Observando-se a Equação \ref{M_inv_equation}, é trivial notar que $u$ será uma combinação linear de $f$ e suas derivadas temporais, sendo os coeficientes fornecidos pela última linha da matriz $M^{-1}$. COm efeito, podemos escrever:
\begin{align}
\label{u_desc}
	u = \beta_0 f + \beta_1 \dot{f} + \beta_2 \ddot{f} + \beta_3 \dddot{f} + \beta_4 f^{(4)}
\end{align}

\paragraph{} O próximo passo, agora, é planejar a trajetória por meio de $f^*(t)$. O modo mais simples é utilizar uma interpolação polinomial, com coeficientes $a_i$ escolhidos de tal forma que o valor de $f^*(t)$ seja nulo até o tempo de início da operação ($t = t'$); também é necessário que $f^*(t)$ atinja um valor constante $c_f$ após um tempo de operação $t_f$ ($t = t' + t_f$). Além disso, é necessário que as cinco primeiras derivadas temporais de $f^*(t)$ sejam nulas nos extremos da operação, de forma a assumir uma trajetória suave de referência para $f^{(4)}$ na vizinhança de $c_f$. Assumindo-se um polinômio de grau 11 para $f^*(t)$, vem
\begin{align}
f^*(t) =
\begin{cases}
0, \forall t < t' \\
\sum_{k = 0}^{11} \! a_{i}t^{k} \\
c_f, \forall t > t_f + t'
\end{cases}
\end{align}

\subsection{Trajetória em Malha Fechada}
\paragraph{} Após o planejamento da trajetória, com a obtenção de $f^*(t)$, o próximo passo é obter uma lei de controle em malha fechada para que se corrija o erro de trajetória dado por $e = f(t) - f^*(t)$. Fazendo uma mudança de coordenadas:
\begin{align}
\label{upsilon_ref}
	\upsilon = -\frac{\beta_0}{\beta_4}f -\frac{\beta_1}{\beta_4}\dot{f} -\frac{\beta_2}{\beta_4}\ddot{f} -\frac{\beta_3}{\beta_4}\dddot{f} + \frac{1}{\beta_4}u
\end{align}

\paragraph{} Tomando-se a Equação \ref{u_desc}, temos que $\upsilon = f^{(4)}$. Com essa informação, podemos configurar um controlador com realimentação \textit{tracking} (Nota: verificar esse conceito!) com a expressão que se segue %TODO Conceito de tracking
\begin{align}
\label{gains_equation}
\upsilon = f^{*(4)} - k_{4}\dddot{e} - k_{3}\ddot{e} - k_{2}\dot{e} - k_{1}e - k_{0}\int_{0}^{t} \! e \mathrm{d}t
\end{align}
\paragraph{} Os valores dos ganhos $k_i$ devem ser tais que o polinômio característico em malha fechada $s^5 + k_{4}s^4 + k_{3}s^3 + k_{2}s^2 + k_{1}s + k_{0}$ seja estável (ver Fabrício et al.). Portanto, o erro $e$ convergirá exponencialmente para 0, e $f(t)$, além de suas derivadas até a 4ª ordem, convergirão para suas trajetórias de referência: $f^*(t) ,..., f^{*(4)}(t)$. Na Equação \ref{gains_equation}, o termo $k_{0}\int_{0}^{t} \! e \mathrm{d}t$ é introduzido de forma a se corrigir erros estáticos causados por perturbações externas.
\paragraph{} A expressão final para a entrada original $u$, portanto, fica
\begin{align}
u = \beta_{0}f + \beta_{1}\dot{f} + \beta_{1}\ddot{f} + \beta_{1}\dddot{f} + \beta_{4}\upsilon
\end{align}
\paragraph{} Em que $\upsilon$ é dado pela Equação \ref{upsilon_ref}.
\end{comment}

\end{document}
	