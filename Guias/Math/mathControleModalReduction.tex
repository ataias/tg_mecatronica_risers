%%%%%%%%%%%%%%%%%%%%%%%%%%%%%%%%%%%%%%%%%
% Short Sectioned Assignment
% LaTeX Template
% Version 1.0 (5/5/12)
%
% This template has been downloaded from:
% http://www.LaTeXTemplates.com
%
% Original author:
% Frits Wenneker (http://www.howtotex.com)
%
% License:
% CC BY-NC-SA 3.0 (http://creativecommons.org/licenses/by-nc-sa/3.0/)
%
%%%%%%%%%%%%%%%%%%%%%%%%%%%%%%%%%%%%%%%%%

%----------------------------------------------------------------------------------------
%	PACKAGES AND OTHER DOCUMENT CONFIGURATIONS
%----------------------------------------------------------------------------------------

\documentclass[a4paper,11pt]{scrartcl} % A4 paper and 11pt font size

\usepackage[T1]{fontenc} % Use 8-bit encoding that has 256 glyphs
\usepackage[utf8]{inputenc}
\usepackage{fourier} % Use the Adobe Utopia font for the document - comment this line to return to the LaTeX default
\usepackage[brazilian]{babel} % English language/hyphenation
\usepackage{amsmath,amsfonts,amsthm} % Math packages

\usepackage{lipsum} % Used for inserting dummy 'Lorem ipsum' text into the template
\usepackage{graphicx}
\usepackage{sectsty} % Allows customizing section commands
\allsectionsfont{\centering \normalfont\scshape} % Make all sections centered, the default font and small caps
\usepackage{float}
\usepackage{fancyhdr} % Custom headers and footers
\pagestyle{fancyplain} % Makes all pages in the document conform to the custom headers and footers
\fancyhead{} % No page header - if you want one, create it in the same way as the footers below
\fancyfoot[L]{} % Empty left footer
\fancyfoot[C]{} % Empty center footer
\fancyfoot[R]{\thepage} % Page numbering for right footer
\renewcommand{\headrulewidth}{0pt} % Remove header underlines
\renewcommand{\footrulewidth}{0pt} % Remove footer underlines
\setlength{\headheight}{13.6pt} % Customize the height of the header

\numberwithin{equation}{section} % Number equations within sections (i.e. 1.1, 1.2, 2.1, 2.2 instead of 1, 2, 3, 4)
\numberwithin{figure}{section} % Number figures within sections (i.e. 1.1, 1.2, 2.1, 2.2 instead of 1, 2, 3, 4)
\numberwithin{table}{section} % Number tables within sections (i.e. 1.1, 1.2, 2.1, 2.2 instead of 1, 2, 3, 4)

\setlength\parindent{0pt} % Removes all indentation from paragraphs - comment this line for an assignment with lots of text

%----------------------------------------------------------------------------------------
%	TITLE SECTION
%----------------------------------------------------------------------------------------

\newcommand{\horrule}[1]{\rule{\linewidth}{#1}} % Create horizontal rule command with 1 argument of height

\title{	
\normalfont \normalsize 
\textsc{Universidade de Brasília} \\ [25pt] % Your university, school and/or department name(s)
\horrule{0.5pt} \\[0.4cm] % Thin top horizontal rule
\huge Implementação de Controle com Redução Modal \\ % The assignment title
\horrule{2pt} \\[0.5cm] % Thick bottom horizontal rule
}

\author{Ataias Pereira Reis \\ Emanuel Pereira Barroso Neto} % Your name

\date{\normalsize\today} % Today's date or a custom date

\usepackage{beramono}
\usepackage{listings}
\usepackage[usenames,dvipsnames]{xcolor}

%%
%% Julia definition (c) 2014 Jubobs
%%
\lstdefinelanguage{Julia}%
  {morekeywords={abstract,break,case,catch,const,continue,do,else,elseif,%
      end,export,false,for,function,immutable,import,importall,if,in,%
      macro,module,otherwise,quote,return,switch,true,try,type,typealias,%
      using,while},%
   sensitive=true,%
   alsoother={\$},%
   morecomment=[l]\#,%
   morecomment=[n]{\#=}{=\#},%
   morestring=[s]{"}{"},%
   morestring=[m]{'}{'},%
}[keywords,comments,strings]%

\lstset{%
    language         = Julia,
    basicstyle       = \ttfamily,
    keywordstyle     = \bfseries\color{blue},
    stringstyle      = \color{magenta},
    commentstyle     = \color{ForestGreen},
    showstringspaces = false,
}


\begin{document}

\maketitle % Print the title

%----------------------------------------------------------------------------------------
%	PROBLEM 1
%----------------------------------------------------------------------------------------

\section{Introdução}
\paragraph{} O objetivo deste documento é apresentar os procedimentos necessários para implementar o método controle apresentado no artigo ``\textit{Modal Reduction Based Tracking Control for Installation of Subsea Equipments}'', desenvolvido por Fabrício et al, em um controlador industrial da Rockwell. Nem todos os detalhes estão presentes no artigo, o que torna difícil simplesmente lê-lo e realizar o sistema.

\section{Equações Governantes}
\paragraph{} Para o riser, a Equação \ref{equacaoMorison} representa o deslocamento horizontal $\Upsilon(z,t)$ do tubo --- um barbante, no caso da bancada de laboratório --- sob a ação de forças hidrodinâmicas externas $F_n(z,t)$ e tração $T(z)$: \begin{align}
	m_s \frac{\partial^2 \Upsilon}{\partial t^2} &= -EJ \frac{\partial^4 \Upsilon}{\partial z^4} + 	\frac{\partial}{\partial z}\left(T(z)\frac{\partial \Upsilon}{\partial z}\right)+F_n(z,t) \label{equacaoMorison}
\end{align}

\paragraph{} Antes de prosseguir, é importante definir termos desta equação: \begin{itemize}
	\item $m_s$ é a massa linear do barbante (o professor Eugênio disse isso, mas o que seria massa linear? É a densidade linear? Ou é simplesmente a massa do barbante?) %TODO Confirmar o que seria massa linear do barbante
	\item $E$ é o módulo de Young do barbante e ele é desconhecido
	\item $J$ é o segundo momento de área e representa a resistência do barbante à flexão. O barbante não apresenta tal resistência, daí $J=0$
	\item $T(z)$ é a força de tração e é dada por \[T(z) = \left(m_b+\frac{z}{L}m_s\right)g,\] sendo $m_b$ a massa da bolinha, $L$ o comprimento do barbante, $z$ a posição vertical sendo o carrinho o zero e $g$ é a força da gravidade. (Aqui, estou considerando $m_s$ como a massa total do barbante. Caso fosse densidade linear, $T(z) = (m_b + z\cdot m_s) g$)
\end{itemize}

\paragraph{} A força externa resultante, $F_n(z,t)$, é dada por \begin{align}
	F_n(z,t) &= -m_f \frac{\partial^2 \Upsilon}{\partial t^2} - \mu \left|\frac{\partial \Upsilon}{\partial t}\right|\frac{\partial \Upsilon}{\partial t}\label{forceN},
\end{align} na qual $\mu$ é o coeficiente de arrasto e $m_f$ é a massa do fluido adicionado, que será posteriormente pormenorizada. Fazendo $m = m_s + m_f$ e substituindo a Equação \ref{forceN} na \ref{equacaoMorison}, obtém-se: \begin{align}
	\frac{\partial^2 \Upsilon}{\partial t^2} &= -\frac{EJ}{m}\frac{\partial^4 \Upsilon}{\partial z^4} + \frac{\partial}{\partial z}\left(\frac{T(z)}{m}\frac{\partial \Upsilon}{\partial z}\right) - \frac{\mu}{m}\left|\frac{\partial \Upsilon}{\partial t}\right|\frac{\partial \Upsilon}{\partial t}
\end{align}


\paragraph{} No artigo do Fabrício, fala ``\textit{Since the traction $T (z)$ is mostly due to the heavy payload,
it can be assumed an average value $T$ for it, taken in the
            middle of the riser’s length.}''. Eu suspeito da veracidade desta frase quando analisado os dados \begin{itemize}
            	\item Massa do isopor: 0.15g
	\item Diâmetro da bolinha: 30.6mm
	\item Massa do barbante: 0.492g
	\item Comprimento do barbante: 0.82m
	\item Diâmetro do barbante: 2mm
	\item Densidade linear considerando densidade volumétrica 191kg/m$^3$ é de 0.6g/m.
\item Massa do barbante pela da bolinha aproximadamente 3 (0.492/0.15).
            \end{itemize}
         
\paragraph{} A massa do barbante, $m_s = 0.492\mathrm{g}$, é bem maior que a massa $m_f$ do fluido adicionado. Para verificar isso, calculemos primeiro a massa $m_{f1}$ do fluido adicionado ao redor do barbante: \begin{align}
	m_{f1} &= 2 V_b \rho_{\mathrm{ar}}\nonumber\\
	&= 2 \pi r^2 L\rho_{\mathrm{ar}}\nonumber\\
	&= 0.025246\mathrm{g}
\end{align} e a massa $m_{f2}$ do fluido adicionado ao redor da bolinha de isopor é \begin{align}
	m_{f2} &= 1.2 V_{e} \rho_{\mathrm{ar}}\nonumber\\
	&= 1.2 \left(\frac{4}{3}\pi r^3\right)\rho_{\mathrm{ar}}\nonumber\\
	&= 4.926\cdot 10^{-5}
\end{align}

\paragraph{} A massa do fluido adicionado é aproximadamente $m_f = m_{f1} + m_{f2} = 0.0253\mathrm{g}$. Tem que a massa do barbante é $m_s/m_f = 19.45$ vezes maior que a massa do fluido adicionado. 
            
\paragraph{} Nota-se que o barbante pesa mais que o isopor, o que faria com que a tração não fosse principalmente devida pela bolinha de isopor, mas sim pelo barbante. Neste caso, não se prossegue usando um valor médio para $T(z)$ como no artigo do Fabrício, mas ainda se define uma constante $\tau$ que substitui o termo $\frac{\mu}{m}\left|\frac{\partial \Upsilon}{\partial t}\right|$ já levando em conta um valor médio para $\left|\frac{\partial \Upsilon}{\partial t}\right|$, resultando em \begin{align}
	\frac{\partial^2 \Upsilon}{\partial t^2} &= -\frac{EJ}{m}\frac{\partial^4 \Upsilon}{\partial z^4} + \frac{\partial}{\partial z}\left(\frac{T(z)}{m}\frac{\partial \Upsilon}{\partial z}\right) - \tau\frac{\partial \Upsilon}{\partial t}\label{EquacaoComTau}
	\end{align}
	
\paragraph{} Antes de prosseguirmos para a discretização, e então obter as matrizes em espaço de estados, é importante pensar nas condições de contorno. No topo, $z=L$, tem-se $\Upsilon(L,t)=u(t)$, ou seja, o carrinho se move conforme uma trajetória $u(t)$ definida. Neste mesmo ponto, $\frac{\partial\Upsilon}{\partial z}(L,t) = 0$. Para a ponta na qual a carga está situada, $z=0$, tem-se $\frac{\partial\Upsilon}{\partial z}(0,t) = \frac{F_L}{T}$, sendo $F_L$ a força aplicada pela ponta do riser na carga. (Outra coisa que confundi, eu entendi $u(t)$ sendo uma trajetória, pois $\Upsilon$ é deslocamento, mas no artigo do Fabrício está escrito em uma momento que é uma força).

\subsection{Discretização}
\paragraph{} De forma a se realizar o controle proposto, o sistema deve ter um espaço de estados finito. Para isso, aplica-se o método de diferenças finitas na coordenada $z$ de maneira a se aproximar a EDP governante em um número finito de EDOs. No espaço discreto, a equação do $k$-ésimo elemento é dada por \begin{align}
	\frac{d^2\Upsilon_k}{dt^2} &= -\frac{EJ}{m l^4}\left(\Upsilon_{k-2} - 4\Upsilon_{k-1}+6\Upsilon_{k}-4\Upsilon_{k+1}+\Upsilon_{k+2}\right)\nonumber\\
	&+ \frac{T_0+mg(k-1)l}{m l^2}\left(\Upsilon_{k-1}-2\Upsilon_{k} + \Upsilon_{k+1}\right)+g\frac{\Upsilon_{k+1}-\Upsilon_{k-1}}{2l}-\tau\frac{d\Upsilon_k}{dt},
\end{align} sendo $l$ a distância entre dois pontos de discretização ($l = L/N$), $L$ é o comprimento do riser e $N$ é o número de pontos de discretização. (A tração $T(z)$ foi trocada por $T$ já na Equação  \ref{EquacaoComTau}, sendo um valor médio).

\paragraph{} Sendo $k\in \mathbb{N}:1\le k \le N$, o que aconteceria quando $k=1$ e se precisasse de $\Upsilon_{k-1}$ e $\Upsilon_{k-2}$? Na realidade, os $N$ pontos se referem aos pontos internos e então este problema não deve ocorrer.

\paragraph{} Para simplificar, definem-se as constantes \begin{align}
	a &= -\frac{EJ}{m l^4}\\
	b_k &= \frac{T_0 + mg(k-1)l}{m l^4}\\
	c_k &= b_k + \frac{g}{2l}
\end{align}

\paragraph{} A meu ver, a melhor forma de se analisar como as matrizes do sistema ficarão é expandir o sistema para casos com $N$ pequeno e ver o que está ocorrendo. Observe que $a=0$ para o barbante, o que simplifica os próximos passos.

\paragraph{} Para o caso $N=6$, tem-se \begin{align}
\mathbf{x} &= \left(\Upsilon_1\;\Upsilon_2\;\Upsilon_3\;\Upsilon_4\;\Upsilon_5\;\Upsilon_6\;\dot{\Upsilon}_1\;\dot{\Upsilon}_2\;\dot{\Upsilon}_3\;\dot{\Upsilon}_4\;\dot{\Upsilon}_5\;\dot{\Upsilon}_6\right)^T 	\\
u &= \Upsilon(L,t) = \Upsilon_7\label{ufor6}\\
y &= \Upsilon(0,t) = \Upsilon_1\label{yfor6}
 \end{align} e as equações são \begin{align}
 	\ddot{\Upsilon}_2 &=  b\left(\Upsilon_{1}-2\Upsilon_{2} + \Upsilon_{3}\right)-\tau \dot{\Upsilon}_2\\
 	\ddot{\Upsilon}_3 &=  b\left(\Upsilon_{2}-2\Upsilon_{3} + \Upsilon_{4}\right)-\tau \dot{\Upsilon}_3\\
 	\ddot{\Upsilon}_4 &=  b\left(\Upsilon_{3}-2\Upsilon_{4} + \Upsilon_{5}\right)-\tau \dot{\Upsilon}_4\\
 	\ddot{\Upsilon}_5 &=  b\left(\Upsilon_{4}-2\Upsilon_{5} + \Upsilon_{6}\right)-\tau \dot{\Upsilon}_5\\
 	\ddot{\Upsilon}_6 &=  b\left(\Upsilon_{5}-2\Upsilon_{6} + u\right)-\tau \dot{\Upsilon}_6.
 \end{align} 
 
 \paragraph{} O que falta fazer é obter a aceleração da bolinha que depende da força da gravidade e da força de Morison. Para isso, considere o pêndulo composto com uma massa na ponta e com a força de Morison atuando contra o movimento da massa na ponta. Para se resolver o problema de obter o ângulo $\theta$ e, posteriormente, a posição horizontal da bolinha, é necessária resolver uma equação diferencial. De modo a obter a equação diferencial, pode-se utilizar a equação que diz que o somatório dos torques é igual ao momento de inércia vezes a aceleração  \begin{align}
 	I\ddot{\theta} &= \sum_{i} M_i
 \end{align}
 
 \paragraph{} Os torques serão calculados a partir do pivô e o sentido anti-horário terá o sinal positivo: \begin{itemize}
 	\item Torque devido pela massa do fio: \begin{align}
 		M_1 &= -m_{\mathrm{fio}} g \frac{L}{2}\sin\theta
 	\end{align}
 	\item Torque devido pela massa da bolinha: \begin{align}
 		M_2 &= -m_{\mathrm{bol}} g L\sin\theta
 	\end{align}
 	\item Torque devido pela força de Morison: \begin{align}
 		M_3 &= \left(m_f\frac{\partial^2 \Upsilon}{\partial t^2}+\mu\left|\frac{\partial \Upsilon}{\partial t}\right|\frac{\partial \Upsilon}{\partial t}\right)L
 	\end{align}
 \end{itemize}
 
 \paragraph{} Assim, teria-se \begin{align}
 	I\frac{\partial^2 \theta}{\partial t^2} &=-m_{\mathrm{fio}} g \frac{L}{2}\sin\theta -m_{\mathrm{bol}} g L\sin\theta + \left(m_f\frac{\partial^2 \Upsilon}{\partial t^2}+\mu\left|\frac{\partial \Upsilon}{\partial t}\right|\frac{\partial \Upsilon}{\partial t}\right)L
 \end{align}
 
% desta forma tem-se \begin{align}
% 	\mathbf{\dot{x}} &= \left[\begin{array}{cccccccccccc}
% 		0 & 0 & 0 & 0 & 0 & 0 & 1 & 0 & 0 & 0 & 0 & 0\\
% 		0 & 0 & 0 & 0 & 0 & 0 & 0 & 1 & 0 & 0 & 0 & 0\\
% 		0 & 0 & 0 & 0 & 0 & 0 & 0 & 0 & 1 & 0 & 0 & 0\\
% 		0 & 0 & 0 & 0 & 0 & 0 & 0 & 0 & 0 & 1 & 0 & 0\\
% 		0 & 0 & 0 & 0 & 0 & 0 & 0 & 0 & 0 & 0 & 1 & 0\\
% 		0 & 0 & 0 & 0 & 0 & 0 & 0 & 0 & 0 & 0 & 0 & 1\\
% 		2b & -5b & 4b & -b & 0 & 0 & -\tau & 0     & 0 & 0 & 0 & 0\\
% 		b & -2b  & b  & 0  & 0 & 0 &  0    & -\tau & 0 & 0 & 0 & 0\\
% 		0 & b & -2b  & b  & 0  & 0 & 0 &  0    & -\tau & 0 & 0 & 0\\
% 		0 & 0 & b & -2b  & b  & 0  & 0 & 0 &  0    & -\tau & 0 & 0\\
% 		0 & 0 & 0 & b & -2b  & b  & 0  & 0 & 0 &  0    & -\tau & 0\\
% 		0 & 0 & 0 & 0 & b & -2b  & 0  & 0 & 0 &  0    & 0   &-\tau\\
% 	\end{array}\right]\mathbf{x} + \left[\begin{array}{c}
%	0\\	0\\	0\\	0\\	0\\ 0\\ 0\\ 0\\ 0\\ 0\\ 0\\ b
%\end{array}
%\right]u
% \end{align} que pode ser representado concisamente como \begin{align}
% 	\mathbf{\dot{x}} &= \left[\begin{array}{cc}
%	\mathbf{0}_{6\times 6} & \mathbf{I}_{6\times 6}\\
%	\mathbf{M}_{6\times 6} & -\tau\mathbf{I}_{6\times 6}\\
%\end{array}\right] \mathbf{x} + \left[\begin{array}{c}
%	\mathbf{0}_{10\times 1}\\ b\\ 2b\\
%\end{array} \right]u
% \end{align}
% 
% \paragraph{} Esta matriz fornece um padrão que pode ser utilizado para se criar um algoritmo que gera as matrizes $\mathbf{A}$ e $\mathbf{B}$ automaticamente, sendo $\mathbf{\dot{x}} = \mathbf{A}\mathbf{x}+\mathbf{B}u$.
% 
% \lstinputlisting[language=Julia,caption={Código para gerar matrizes A, B e C},label=generateABC]{code/GenerateABC.jl}

%\section{Redução Modal}
\end{document}
