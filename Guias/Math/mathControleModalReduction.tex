%%%%%%%%%%%%%%%%%%%%%%%%%%%%%%%%%%%%%%%%%
% Short Sectioned Assignment
% LaTeX Template
% Version 1.0 (5/5/12)
%
% This template has been downloaded from:
% http://www.LaTeXTemplates.com
%
% Original author:
% Frits Wenneker (http://www.howtotex.com)
%
% License:
% CC BY-NC-SA 3.0 (http://creativecommons.org/licenses/by-nc-sa/3.0/)
%
%%%%%%%%%%%%%%%%%%%%%%%%%%%%%%%%%%%%%%%%%

%----------------------------------------------------------------------------------------
%	PACKAGES AND OTHER DOCUMENT CONFIGURATIONS
%----------------------------------------------------------------------------------------

\documentclass[a4paper,11pt]{scrartcl} % A4 paper and 11pt font size

\usepackage[T1]{fontenc} % Use 8-bit encoding that has 256 glyphs
\usepackage[utf8]{inputenc}
\usepackage{fourier} % Use the Adobe Utopia font for the document - comment this line to return to the LaTeX default
\usepackage[brazilian]{babel} % English language/hyphenation
\usepackage{amsmath,amsfonts,amsthm} % Math packages

\usepackage{lipsum} % Used for inserting dummy 'Lorem ipsum' text into the template
\usepackage{graphicx}
\usepackage{sectsty} % Allows customizing section commands
\allsectionsfont{\centering \normalfont\scshape} % Make all sections centered, the default font and small caps
\usepackage{float}
\usepackage{fancyhdr} % Custom headers and footers
\pagestyle{fancyplain} % Makes all pages in the document conform to the custom headers and footers
\fancyhead{} % No page header - if you want one, create it in the same way as the footers below
\fancyfoot[L]{} % Empty left footer
\fancyfoot[C]{} % Empty center footer
\fancyfoot[R]{\thepage} % Page numbering for right footer
\renewcommand{\headrulewidth}{0pt} % Remove header underlines
\renewcommand{\footrulewidth}{0pt} % Remove footer underlines
\setlength{\headheight}{13.6pt} % Customize the height of the header

\numberwithin{equation}{section} % Number equations within sections (i.e. 1.1, 1.2, 2.1, 2.2 instead of 1, 2, 3, 4)
\numberwithin{figure}{section} % Number figures within sections (i.e. 1.1, 1.2, 2.1, 2.2 instead of 1, 2, 3, 4)
\numberwithin{table}{section} % Number tables within sections (i.e. 1.1, 1.2, 2.1, 2.2 instead of 1, 2, 3, 4)

\setlength\parindent{0pt} % Removes all indentation from paragraphs - comment this line for an assignment with lots of text

%----------------------------------------------------------------------------------------
%	TITLE SECTION
%----------------------------------------------------------------------------------------

\newcommand{\horrule}[1]{\rule{\linewidth}{#1}} % Create horizontal rule command with 1 argument of height

\title{	
\normalfont \normalsize 
\textsc{Universidade de Brasília} \\ [25pt] % Your university, school and/or department name(s)
\horrule{0.5pt} \\[0.4cm] % Thin top horizontal rule
\huge Implementação de Controle com Redução Modal \\ % The assignment title
\horrule{2pt} \\[0.5cm] % Thick bottom horizontal rule
}

\author{Ataias Pereira Reis \\ Emanuel Pereira Barroso Neto} % Your name

\date{\normalsize\today} % Today's date or a custom date

\begin{document}

\maketitle % Print the title

%----------------------------------------------------------------------------------------
%	PROBLEM 1
%----------------------------------------------------------------------------------------

\section{Introdução}
\paragraph{} O objetivo deste documento é de apresentar todos os procedimentos necessários para implementar o controle apresentado no artigo ``\textit{Modal Reduction Based Tracking Control for Installation of Subsea Equipments}'', desenvolvido por Fabrício et al, em um controlador industrial da Rockwell. Para alguém iniciante no assunto, é difícil só ler o artigo e realizar a implementação diretamente.

\section{Equações Governantes}
\paragraph{} Para o riser, a Equação \ref{equacaoMorison} representa o deslocamento horizontal $\Upsilon(z,t)$ do tubo --- um barbante, no presente caso --- sob a ação de forças hidrodinâmicas externas e tração: \begin{align}
	m_s \frac{\partial^2 \Upsilon}{\partial t^2} &= -EJ \frac{\partial^4 \Upsilon}{\partial z^4} + 	\frac{\partial}{\partial z}\left(T(z)\frac{\partial \Upsilon}{\partial z}\right)+F_n(z,t) \label{equacaoMorison}
\end{align}

\paragraph{} Antes de prosseguir, é importante definir termos desta equação: \begin{itemize}
	\item $m_s$ é a massa linear do barbante (o professor Eugênio disse isso, mas o que seria massa linear? É a densidade do linear? Ou é simplesmente a massa do barbante?) %TODO Confirmar o que seria massa linear do barbante
	\item $E$ é o módulo de Young do barbante e ele é desconhecido
	\item $J$ é o segundo momento de área e representa a resistência do barbante à flexão. O barbante não apresenta tal resistência, daí $J=0$
	\item $T(z)$ é a força de tração e é dada por \[T(z) = \left(m_b+\frac{z}{L}m_s\right)g,\] sendo $m_b$ a massa da bolinha, $L$ o comprimento do barbante, $z$ a posição vertical sendo o carrinho o zero e $g$ é a força da gravidade. (Aqui, estou considerando $m_s$ como a massa total do barbante)
\end{itemize}

\paragraph{} A força externa resultante, $F_n(z,t)$, é dada por \begin{align}
	F_n(z,t) &= -m_f \frac{\partial^2 \Upsilon}{\partial t^2} - \mu \left|\frac{\partial \Upsilon}{\partial t}\right|\frac{\partial \Upsilon}{\partial t}\label{forceN},
\end{align} na qual $\mu$ é o coeficiente de arrasto e $m_f$ é a massa do fluido adicionado, que será posteriormente pormenorizada. Fazendo $m = m_s + m_f$ e substituindo a Equação \ref{forceN} na \ref{equacaoMorison}, obtém-se: \begin{align}
	\frac{\partial^2 \Upsilon}{\partial t^2} &= -\frac{EJ}{m}\frac{\partial^4 \Upsilon}{\partial z^4} + \frac{\partial}{\partial z}\left(\frac{T(z)}{m}\frac{\partial \Upsilon}{\partial z}\right) - \frac{\mu}{m}\left|\frac{\partial \Upsilon}{\partial t}\right|\frac{\partial \Upsilon}{\partial t}
\end{align}
\paragraph{} No artigo do Fabrício, fala ``\textit{Since the traction $T (z)$ is mostly due to the heavy payload,
it can be assumed an average value $T$ for it, taken in the
            middle of the riser’s length.}''. Eu suspeito da veracidade desta frase quando analisado os dados \begin{itemize}
            	\item Massa do isopor: 0.15g
	\item Diâmetro da bolinha: 30.6mm
	\item Massa do barbante: 0.492g
	\item Comprimento do barbante: 0.82m
	\item Diâmetro do barbante: 2mm
	\item Densidade linear considerando densidade volumétrica 191kg/m$^3$ é de 0.6g/m.
\item Massa do barbante pela da bolinha aproximadamente 3 (0.492/0.15).
            \end{itemize}
            
\paragraph{} Nota-se que o barbante pesa mais que o isopor, o que faria com que a tração não fosse principalmente devida pela bolinha, mas sim pelo isopor. Neste caso, pode-se prosseguir com o uso de um valor médio para $T(z)$? No momento considero que sim, pois acho que foi assim que foi feito nas simulações. Definindo uma constante $\tau$ que substitui o termo $\frac{\mu}{m}\left|\frac{\partial \Upsilon}{\partial t}\right|$ já levando em conta um valor médio para $\left|\frac{\partial \Upsilon}{\partial t}\right|$ e usando o valor médio $T$ para a tração, tem-se \begin{align}
	\frac{\partial^2 \Upsilon}{\partial t^2} &= -\frac{EJ}{m}\frac{\partial^4 \Upsilon}{\partial z^4} + \frac{T}{m}\frac{\partial^2 \Upsilon}{\partial z^2} - \tau\frac{\partial \Upsilon}{\partial t}\label{EquacaoComTau}
	\end{align}
	
\paragraph{} Antes de prosseguirmos para a discretização, e então obter as matrizes em espaço de estados, é importante pensar nas condições de contorno. No topo, $z=L$, tem-se $\Upsilon(L,t)=u(t)$, ou seja, o carrinho se move conforme uma trajetória $u(t)$ definida. Neste mesmo ponto, $\frac{\partial\Upsilon}{\partial z}(L,t) = 0$. Para a ponta na qual a carga está situada, $z=0$, tem-se $\frac{\partial\Upsilon}{\partial z}(0,t) = \frac{F_L}{T}$, sendo $F_L$ a força aplicada pela ponta do riser na carga. (Outra coisa que confundi, eu entendi $u(t)$ sendo uma trajetória, pois $\Upsilon$ é deslocamento, mas no artigo do Fabrício está escrito em uma momento que é uma força).

\subsection{Discretização}
\paragraph{} De forma a se realizar o controle proposto, o sistema deve ter um espaço de estados finito. Para isso, aplica-se o método de diferenças finitas na coordenada $z$, de maneira a se aproximar a EDP governante em um número finito de EDOs. No espaço discreto, a equação do $k$-ésimo elemento é dada por \begin{align}
	\frac{d^2\Upsilon}{dt^2} &= -\frac{EJ}{m\Delta z^4}\left(\Upsilon_{k-2} - 4\Upsilon_{k-1}+6\Upsilon_{k}-4\Upsilon_{k+1}+\Upsilon_{k+2}\right)\nonumber\\
	&+ \frac{T}{m\Delta z^2}\left(\Upsilon_{k-1}-2\Upsilon_{k} + \Upsilon_{k+1}\right)-\tau\frac{d\Upsilon}{dt},
\end{align} sendo $\Delta z$ a distância entre dois pontos de discretização ($\Delta z = L/N$), $L$ é o comprimento do riser e $N$ é o número de pontos de discretização. (A tração $T(z)$ foi trocada por $T$ já na Equação  \ref{EquacaoComTau}, sendo um valor médio).

\paragraph{} Sendo $k\in \mathbb{N}:1\le k \le N$, o que aconteceria quando $k=1$ e se precisasse de $\Upsilon_{k-1}$ e $\Upsilon_{k-2}$? Estes termos estariam fora do domínio. Para esse caso especial, deve ser utilizado diferenças finitas unilaterais, ou basta essa única discretização? Um problema similar ocorrer quando $k=N$.

\section{Redução Modal}
\end{document}
